
%\ aquí se menciona el proposito, descripción del producto
\addcontentsline{toc}{chapter}{Introducción}
\chapter*{Introducción}

\stepcounter{chapter}
Una vez que es completada los requerimientos del proyecto y los objetivos que es el desarrollo del Sistema Software de Gestión de Farmacia. Con tiempo y presupuesto limitados. En aras de la satisfacción de la cliente se reunió con todos los miembros del proyecto para estructurar los puntos principales y asignación de roles. 

\section{Descripción del proyecto}
El usuario o cliente del sistema de gestión de farmacia es la dueña de la farmacia llamada farmacia “Kumara”. El nombre de la dueña es Reina Santi. Esta farmacia inicio hace años siendo primeramente una tienda pequeña que fue creciendo con el pasar de los años y actualmente tiene alrededor de 5 sucursales en la ciudad de La Paz, por lo que se hace difícil seguir manejando la información de manera manual. Actualmente se tiene registrado todo en hojas de archivos Excel desde los datos de los medicamentos, los empleados (farmacéuticos), los proveedores, las compras y las ventas. Con el pasar de los años la farmacia se hizo de reputación y también tiene convenios con ciertos hospitales para dar medicamentos prestados o con cierto porcentaje de descuento. Por normativa de impuestos nacionales ordenaron que todos los negocios mucho más siendo esta una cadena de farmacia deben de tener un sistema de facturación computarizada por lo que se requiere que el sistema además de las ventas maneje lo que es la facturación con las nuevas disposiciones de impuestos internos que son el código de control y ahora el nuevo código QR. 
\section{Necesidades del proyecto}
La señora Reina Santi quiere tener la posibilidad de poder ver toda la gestión de todas sus cadenas de farmacia, ella tiene que poder ver todo por sucursal y en general. Sus farmacéuticos solo pueden tener información de inventarios de los medicamentos, la ubicación y desde luego los precios. El cajero solo puede ver información pertinente a las ventas y facturación, él no puede ver inventarios de medicamentos ni ubicación ni stock. También se requiere que el sistema tenga notificaciones sobre el inventario de los medicamentos para ver que se estén quedando sin un determinado medicamento. Las alertas se las debe enviar cuando se tenga 20 unidades de cierto medicamento. Desde luego que el sistema debe emitir reportes y consultas correspondientes a ventas, inventarios, stock, farmacéuticos, medicamentos, etc.
%\section{Abreviaturas}
\section{Propósito u objetivo}
El objetivo es incrementar la satisfacción de los clientes en un 80\% de la farmacia, eliminar el uso de hojas de archivos Excel para los datos de los medicamentos, los empleados (farmacéuticos), los proveedores, las compras y las ventas. Además cumplir con la normativa de impuestos nacionales
\section{Alcance}
El proyecto se desarrollará en 3 meses desde el día de inicio. Se realizará según las necesidades del proyecto, de acuerdo al costo,  tiempo y definición de requerimientos. Gestionará la cadena de farmacias y emitirá reportes de ventas, inventarios, stock, farmacéuticos y medicamentos\cite{ahmed_software_2012}.  Ademas implementar un sistema de facturación computarizada



\addcontentsline{toc}{chapter}{Vista general del proyecto}
\chapter*{Vista general del proyecto}
\stepcounter{chapter}
%Guidance
%This paragraph shall briefly state the purpose of the system and the software to which this
%document applies. It shall describe the general nature of the system and software; summarize the
%history of system development, operation, and maintenance; identify the project sponsor,
%acquirer, user, developer, and support agencies; identify current and planned operating sites;
%and list other relevant documents. 
El proyecto de FARMAKUM incluye el desarrollo de software AdmiFarm, un sistema de inventario y facturación alojado en la nube. Admifarm provee los requisitos de comunicación segura

\section{Suposiciones y restricciones}

%Pasos para control de proyecto
\addcontentsline{toc}{chapter}{Organización del Proyecto}
\chapter*{Organización del Proyecto}
\stepcounter{chapter}
\section{Participantes en el proyecto}
%https://monday1006.monday.com/boards/3478543214  aquí se puede hacer una tabla
\subsection{Jefe de proyecto}
\subsection{Analista de sistemas}
\subsection{Desarrolladores}
\subsection{Tester}
\section{Roles y responsabilidades}
\begin{table}
%\centering
%\begin{tabular}{|Posición|Responsabilidades o rol|}\hline
% Patrocinador & \\
% Equitpo de desarrollo & 
%\end{tabular}
\end{table}
\addcontentsline{toc}{chapter}{Gestión de proceso}
\chapter*{Gestión de proceso}
\stepcounter{chapter}
%en ingles gestion es management
\section{Estimación del proyecto}
%Faces de desarrollo
%Objetivos
%Plan de despliegue o liberación
\subsection{Estimación de recursos humanos}
\subsection{Estimación de recursos de software}
%\subsection{Estimación del tiempo}
\section{Plan de proyecto}
\subsubsection{Calendario de proyecto}
\subsubsection{Diagrama de Gantt}
%https://www.teamgantt.com/

\section{Seguimiento y control de proyecto}
%\chapter*{Documentación}

